\section{Conclusioni}
\subsection{Rispetto delle specifiche}
Per minimizzare il consumo energetico dei nodi sensore il sistema
prevede che i moteOBJ in modalit\`a di funzionamento standard tengano
accesa l'interfaccia radio per una finestra temporale di durata
limitata, intervallata a periodi di attesa. I moteOBJ usano
l'interfaccia radio in modo attivo solo se sollecitati da un pacchetto
inviato da un motePC. La trasmissione delle SIFT \`e incrementale, in
modo da poter minimizzare l'utilizzo dell'interfaccia radio nel caso
che il matching non richieda tutte le SIFT che il tmote
possiede. Infine quando il PC sceglie un tmote per richiedere le sue
SIFT gli altri tmote vengono spenti con un pacchetto SLEEP per evitare
che siano inutilmente attivi.

Gli eventi missed detection e false detection sono unicamente
imputabili al sistema
di matching utilizzato, come gi\`a spiegato le SIFT rappresentano un
descrittore particolarmente robusto nei confronti di questi eventi,
anche se, come i tutti gli algoritmi di matching basati su periferiche di
visione, il risultato \`e molto influenzato dall'illuminazione
dell'ambiente circostante.

La velocit\`a di spostamento consentita all'agente mobile \`e limitata
dalla complessit\`a computazionale richiesta dall'algoritmo di
matching, e dalla necessaria nitidezza dell'immagine acquisita con la
camera.

\subsection{Risultati ottenuti}
Il sistema \`e stato testato con tre oggetti diversi ed un pc atto
alla ricerca. Son stati analizzati in dettaglio tutti i passaggi dei
vari programmi e verificati i funzionamenti di tutti i tipi di
pacchetto. 

Sia la parte relativa al protocollo di comunicazione implementata sui
sensori, sia il programma di {\em matching} hanno dato buoni
risultati. 

\subsection{Considerazioni finali}
Il punto di forza dello sviluppo risiede sicuramente nel protocollo,
ideato e strutturato ad-hoc per il progetto. Implementa tutte
delle necessit\`a emerse nel corso dello sviluppo. La struttura dello
stesso \`e indipendente dall tipo di dati che vengono
scambiati. Agendo solamente sul SIFT ID e sul sequence number \`e
possibile immagazzinare e poi interagire con qualsiasi struttura dati.

La robustezza del matching attraverso le sift richiede per\`o un
discreto costo computazionale, forse eccessivo per l'utilizzo in
agenti mobili dotati di CPU a clock ridotto.

Per quanto riguarda il consumo energetico del sistema, la nostra
implementazione si \`e concentrata nell'ottimizzazione dell'utilizzo
dell'interfaccia radio, rendendo parametrici gli intervalli di sleep e
discover in maniera che possano essere impostati a seconda dello
scenario.
