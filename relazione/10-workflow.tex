\section{Workflow}
% Workflow teorico del processo
L'impostazione teorica del processo di digitalizzazione dell'informazione sonora contenuta nei dischi shellac si articola su quattro fasi principali:
\begin{itemize}
	\item scansione
	\item image processing
	\item sound extraction
	\item filtering
\end{itemize}
\begin{figure}[h!t]
\begin{center}
\includegraphics[scale=0.2]{./img/block-scheme.png}
\caption{Workflow generale}
\end{center}
\end{figure}
\subsection{Shellac}
I dischi in gommalacca, seppur soggetti a variazioni dovute ai diversi produttori e periodi di produzione, evidenziano delle caratteristiche il pi\`u delle volte comuni:
\begin{itemize}
	\item diametro tra gli 8 e 12 pollici
	\item massima escursione possibile del solco: 0.15 mm
	\item banda teorica: da 30 Hz 1.6 kHz
	\item banda reale: da 500 Hz a 3.5 kHz
\end{itemize}
Un primo problema che affligge il processo in questione pu\`o, anche intuitivamente, essere dedotto dall'osservazione della massima escursione possibile del solco: 0.15 mm, in poco pi\`u di un decimo di millimetro \`e infatti contenuta tutta l'informazione sonora che si vuole digitalizzare, di conseguenza la risoluzione dell'immagine ottenuta attraverso il processo di scansione dev'essere sufficientemente alta da poter recuperare il preciso andamento dei solchi.

\subsection{Scansione}
La fase di scansione permette di ottenere l'immagine delle tracce che verr\`a in seguito processata.
\begin{figure}[h!t]
\begin{center}
\includegraphics[scale=0.6]{./img/shellac-track.png}
\caption{Porzione di tracce}
\end{center}
\end{figure}
Ottenere un'immagine sufficientemente dettagliata da essere usata negli step successivi di elaborazione si \`e rivelata la fase pi\`u aleatoria dell'intero processo, il livello qualitativo \`e infatti inficiato da svariati fattori derivanti dallo scanner utilizzato e dal posizionamento del disco stesso sul piano del dispositivo.

Dai risultati sperimentali di questa fase \`e emersa la necessit\`a di utilizzare una risoluzione di scansione di 2400 dpi ottici (cio\`e non derivanti da interpolazione software operata dal driver dello scanner) e di rilevare - per ogni scanner utilizzato - il punto di illuminazione ottimale al fine di determinare la porzione di informazione contenente la maggior quantit\`a possibile di informazione.

Il punto di illuminazione ottimale \`e risultato ampiamente variabile a seconda del modello di scanner utilizzato determinando, di conseguenza, la necessit\`a di intraprendere un processo di calibrazione tutte le volte che risulti necessario cambiare dispositivo.

\subsection{Image processing}
La fase di image processing si articola in tre passaggi fondamentali:
\begin{itemize}
	\item ricerca del centro
	\item unwrap dell'immagine
	\item crop dell'immagine
\end{itemize}

\subsubsection{Ricerca del centro}
\begin{figure}[h!t]
\begin{center}
\includegraphics[scale=0.15]{./img/center.png}
\caption{Ricerca del centro}
\end{center}
\end{figure}
La determinazione del centro del disco scannerizzato \`e necessaria alla successiva trasformazione da coordinate cartesiane a coordinate polari che permette di ``raddrizzare'' le tracce.

La detection del centro ha inizio con la determinazione dei bordi del disco sulla base della variazione di luminosit\`a tra lo sfondo dell'immagine (bianco) e la porzione di disco acquisita tramite scansione (sensibilmente pi\`u scura).
Una volta acquisito un numero $N$ di coppie di punti $(x_i,y_i)$ giacenti sul bordo del disco, siano $(x_c, y_c)$ le coordinate del centro e $r$ il raggio del disco, si procede alla soluzione dell'equazione:
$$(x_i-x_c)^2+(y_i-y_c)^2=r^2 \quad i=1,2,\ldots,N$$
che pu\`o essere riscritta come:
$$r^2-x_c^2-y_c^2+2x_cx_i+2y_cy_i = x_i^2+y_i^2$$
e ponendo:
$$
\begin{cases}
a_1= r^2 - x_c^2 - y_c^2\\
a_2 = 2x_c\\
a_3 = 2y_c\\
\end{cases}
$$
si ottiene la forma matriciale:
$$
\begin{pmatrix}
1 && x_1 && y_1\\
1 && x_2 && y_2\\
\vdots && \vdots && \vdots\\
1 && x_N && y_N
\end{pmatrix}
\begin{pmatrix}
a_1 \\
a_2 \\
a_3
\end{pmatrix}
$$
$$
=
\begin{pmatrix}
x_1^2+y_1^2\\
x_2^2+y_2^2\\
\vdots \\
x_N^2+y_N^2
\end{pmatrix}
$$
che, risolta per mezzo del metodo di eliminazione di Gauss-Jordan, fornisce i parametri della circonferenza esterna del disco in esame:
$$
\begin{cases}
	x_c ={a_2\over{2}} \\
	y_c = {a_3\over{2}}\\
	r = \sqrt{x_c^2+y_c^2 + a_3}\\
\end{cases}
$$
\subsubsection{Unwrap}
Una volta determinato il centro del disco, si procede al ``raddrizzamento'' delle tracce per mezzo della trasformazione da coordinate cartesiane a polari:
\begin{figure}[h!t]
\begin{center}
\includegraphics[scale=0.5]{./img/cartesio.png}
\caption{Rappresentazione in coordinate cartesiane}\label{cartesio}
\end{center}
\end{figure}

\begin{figure}[h!t]
\begin{center}
\includegraphics[scale=0.5]{./img/polare.png}
\caption{Rappresentazione in coordinate polari}
\end{center}
\end{figure}
Sia $v(x,y)$ un punto dell'immagine in coordinate cartesiane, la sua trasformazione in coordinate polari sar\`a: $u(r,\theta) = v(x_c+rcos(\theta), y_c+rsin(\theta))$, dove $r$ e $\theta$ sono definiti in figura \ref{cartesio}
\subsubsection{Crop}
Il cropping elimina parti non necessarie dell'immagine, come per esempio quelle relative all'etichetta, in modo da ottenere un file di
\begin{figure}[h!t]
\begin{center}
\includegraphics[scale=0.6]{./img/cropping.png}
\caption{Esempio di cropping}
\end{center}
\end{figure}
 dimensioni minori contenente solo l'informazione necessaria al processo.
\subsection{Sound extraction}

\subsubsection{Ricerca delle tracce}
Una volta determinato il centro e croppata l'immagine si prosegue alla localizzazione delle tracce ed al loro inseguimento.

Ogni immagine derivante da scansione viene memorizzata in forma matriciale 
$$I(h,w) = \begin{pmatrix} 
	I(1,1) && \ldots && I(1,w) \\
	\vdots && \vdots && \vdots \\
	I(h,1) && \ldots && I(h,w)
\end{pmatrix}$$  $$I(h,w) \in \{0, 1,2,\ldots, 255\}$$
L'algoritmo di localizzazione fa affidamento sul fatto che i pixel appartenenti a tracce risultino molto pi\`u luminosi rispetto agli altri, quindi sommando i pixel relativi alle righe di $I(h,w)$ (facendo quindi variare l'indice di colonna) si avranno dei picchi in corrispondenza delle tracce (righe).
\begin{figure}[h!t]
\begin{center}
\includegraphics[scale=0.3]{./img/track-detection.png}
\caption{Picchi generati dall'algoritmo di localizzazione}
\end{center}
\end{figure}
\subsubsection{Inseguimeno delle tracce}
La conoscenza dei punti iniziali delle tracce nell'immagine scansionata permette il loro inseguimento per mezzo di un algoritmo euristico ad hoc.

Innanzitutto un vettore di 5 pixel di altezza viene centrato sul punto iniziale della traccia da seguire:
$$
P = 
\begin{pmatrix}
M(y_1+r(0), c)\\
\vdots \\
M(y+1+r(4), c)
\end{pmatrix}
$$
dove $r(i) \in \{-2,-1,0,1,2\}$ e c \`e l'indice di colonna.

Questi valori sono poi usati come pesi per il calcolo del centro di massa del vettore (corrispondente alla posizione della traccia relativamente alla colonna in cui quest'ultimo \`e stato posto): $\rho(n) = {\sum_{i=1}^{5}{P(i)r(i)}\over{\sum_{i=1}^{5}{P(i)c}}}$
\begin{figure}[h!t]
\begin{center}
\includegraphics[scale=0.5]{./img/track-following.png}
\caption{Inseguimento delle tracce}
\end{center}
\end{figure}
Ad ogni passo dell'algoritmo il vettore di 5 pixel viene spostato in avanti di una colonna e ne viene determinato il centrodi massa.
\subsubsection{Concatenazione delle tracce}
\begin{figure}[h!t]
\begin{center}
\includegraphics[scale=0.5]{./img/concatenation.png}
\caption{Concatenazione delle tracce}\label{fading}
\end{center}
\end{figure}
Una volta ottenuti gli spezzoni ti tracce da ogni singola immagine, questi devono essere concatenati in modo da ricostruire il brano originale. Una spiegazione grafica di questo processo \`e offerta in figura \ref{fading}
\section{Il progetto analizzato}
L'analisi del progetto sviluppato presso la Princeton University da Mark McCann, Paul Calamia e Nir Ailon si \`e articolata in quattro fasi:
\begin{itemize}
\item Analisi del codice Matlab
\item Creazione del flusso di lavoro
\item Associazione tra flusso del codice e modello teorico
\item Debugging e refactoring del codice
\end{itemize}
\subsection{Analisi del codice ed estrapolazione del flusso di lavoro}
%%manca la parte sul matlab