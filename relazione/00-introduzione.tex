\section{Introduzione}
% sezione introduttiva, consegna, scopo
L'obiettivo del progetto \`e la verifica dello stato dell'arte
nell'estrazione di segnale audio da una scansione di dischi
in caralacca (shellac).

% XXX Fa sempre bene ricordarsi come si inserisce un'immagine
% 
% \begin{figure}[h!t]
% \begin{center}
% \includegraphics[scale=0.3]{../img/schema_generale.pdf}
% \caption{Schema generale di funzionamento}
% \end{center}
% \end{figure}
% 
