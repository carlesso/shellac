\begin{frame}
\frametitle{La realtà di interesse}
\begin{block}{Un approccio sistematico}
Qui ci và uno schema a blocchi 
\end{block}
\end{frame}

\begin{frame}
\frametitle{La realtà di interesse}
\begin{block}{Il processo di scansione : considerazioni preliminari e personali}
Analisi preventiva per capire i problemi da affrontare:
\begin{itemize}
\item[*] Scegliere la risoluzione adatta (test fatti con matlab quantizz.);
\item[*] \`Individuare la porzione di disco meglio illuminata (discorsi su ottiche);
\item[*] Come partizionare il disco?
\end{itemize}
\end{block}

\begin{block}{Il processo di scansione : cosa suggerisce la teoria?}
Studio della letteratura per discriminare tra possibili scelte 
\begin{itemize}
\item[*] Risoluzione 2400 dpi ottici;
\item[*] Non sempre banale la scelta della posizione corretta per la scansione;
\item[*] Scelte per il partizionamento varie (solitamente 4 fette).
\end{itemize}
\end{block}
\end{frame}

\begin{frame}
\frametitle{La realtà di interesse}
\begin{block}{L'elaborazione dell'immagine : cosa suggerisce la teoria}
Individuate 3 fasi principali
\begin{itemize}
\item[*] Ricerca del centro;
\item[*] Passaggio da coordinate cartesiane a polari(la porzione viene spalmata);
\item[*] Cropping dell'immagine.
\end{itemize}
\end{block}
\end{frame}

\begin{frame}
\frametitle{La realtà di interesse}
\begin{block}{L'estrazione del suono : considerazioni preliminari}
\`E stato necessario capire molto bene come mappare le variazioni presenti
sul disco, in mod da effettuare una traduzione fedele.
\end{block}
\begin{block}{L'estrazione del suono : cosa suggerisce la teoria}
\begin{itemize}
\item[*] Ricerca delle traccie;
\item[*] Inseguimento delle traccie;
\item[*] Concatenzione delle traccie.
\end{itemize}
\end{block}
\end{frame}
