\begin{frame}
\frametitle{La realtà di interesse}
\begin{block}{Un approccio sistematico}
Qui ci và uno schema a blocchi 
\end{block}
\end{frame}

\begin{frame}
\frametitle{La realtà di interesse}
\begin{block}{Il processo di scansione: considerazioni preliminari e personali}
Analisi preventiva per capire i problemi da affrontare:
\begin{itemize}
\item[*] scegliere la risoluzione adatta;
\item[*] individuare la porzione della scansione con più informazione;
%\item[*] individuare la porzione di disco meglio illuminata (discorsi su ottiche);
\item[*] come partizionare il disco?
\end{itemize}
\end{block}

\begin{block}{Il processo di scansione: cosa suggerisce la teoria?}
%Studio della letteratura per discriminare tra possibili scelte 
\begin{itemize}
\item[*] risoluzione 2400 dpi ottici;
\item[*] non sempre banale la scelta della posizione corretta per la scansione;
\item[*] scelte per il partizionamento varie (solitamente 4 fette).
\end{itemize}
\end{block}
\end{frame}

\begin{frame}
\frametitle{La realtà di interesse}
\begin{block}{L'elaborazione dell'immagine: cosa suggerisce la teoria}
Individuate tre fasi principali
\begin{itemize}
\item[*] ricerca del centro;
\item[*] unwrap dell'immagine;
%\item[*] Passaggio da coordinate cartesiane a polari(la porzione viene spalmata);
\item[*] crop dell'immagine.
\end{itemize}
\end{block}
\end{frame}

\begin{frame}
\frametitle{La realtà di interesse}
\begin{block}{L'estrazione del suono: considerazioni preliminari}
\'E stato necessario capire molto bene come mappare le variazioni presenti
sul disco, in modo da effettuare una traduzione sensata e fedele.
\end{block}
\begin{block}{L'estrazione del suono: cosa suggerisce la teoria}
\begin{itemize}
\item[*] ricerca delle traccie;
\item[*] inseguimento delle traccie;
\item[*] concatenzione delle traccie.
\end{itemize}
\end{block}
\end{frame}
