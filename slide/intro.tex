\begin{frame}

\frametitle{Obbiettivi del progetto}

\begin{block}{Cosa?}
%inserire immagine disco con l'eticheta ---> scanner 
Estrarre la traccia audio impressa in un disco in \textbf{gommalacca}
(\emph{shellac}) dalla scansione ottica del disco.
%Ricavare le informazioni audio, presenti in un disco di \textbf{ceralacca}
%(\emph{shellac}) a 78 giri, utilizzando strumenti ottici, in particolare scanner.
\end{block}

\begin{block}{Perché?}
\begin{itemize}
\item La riproduzione dei dischi \emph{shellac} avviene attraverso il \textbf{grammofono}:
\begin{itemize}
\item Deterioramento del \emph{groove} a causa della puntina
\item Apparecchio ad alta obsolescenza: \textbf{delicato}
\end{itemize}
\item Frequente assenza dei \emph{master}, diversamente dai vinili
\item Processo acquisizione informazione automatizzabile
\item \textbf{giradischi} costosi e complessi da usare (formazione personale)
\end{itemize}
\end{block}

\end{frame}
