\begin{frame}

\frametitle{Obbiettivi del progetto}

\begin{block}{Cosa?}
%inserire immagine disco con l'eticheta ---> scanner 
Estrapolare la traccia audio impressa in un disco in \textbf{ceralacca}
(\emph{shellac}) dalla scansione ottica del disco.
%Ricavare le informazioni audio, presenti in un disco di \textbf{ceralacca}
%(\emph{shellac}) a 78 giri, utilizzando strumenti ottici, in particolare scanner.
\end{block}

\begin{block}{Perché?}
\begin{itemize}
\item[*]i dischi \emph{shellac} tendono a venire deteriorati a causa del
scorrere della puntina all'interno del \emph{groove};
\item[*]a differenza dei dischi in vinile spesso non  esistono i \emph{master};
\item[*]per amore della scienza?
%\item[*]L'idea di leggere un disco in ceralacca con uno scanner, permette
%di prevenire la degradazione del supporto, usando un dispositivo poco costoso
%e comune!
\end{itemize}
\end{block}

\end{frame}
