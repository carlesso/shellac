\begin{frame}

\frametitle{La realtà di interesse}

\begin{block}{Lo Shellac}
\`E interessante conoscere alcune caratteristiche di questi oggetti:
\begin{itemize}
\item[*]Diametro del disco = 25 [cm.]
\item[*]Massima escursione possibile del solco = $\sim$ 0.15 [mm.]
\item[*]Banda utile = [30Hz. $\div$ 16000Hz.]
\end{itemize}
Il segnale è modulato lateralemente $\Rightarrow$ Un semplice scanner è 
sufficente per ricavare le forme d'onda
\end{block}

\begin{block}{Il grammofono}
La riproduzione dei dischi \emph{shellac} avviene attraverso il \textbf{Grammofono},
che, almeno agli inizi della sua storia, prevedeva trasduzione da solco
a suono, meramente meccanica.\\
La diffusione dei grammofoni è andata scemando con l'avvento di nuovi supporti
e di nuovi strumenti di lettura e ciò li ha resi oggetti \textbf{costosi}
e \textbf{difficili da reperire}.
\end{block}

\begin{frame}
\begin{block}{Un approccio sistematico}

\end{block}

\end{frame}
