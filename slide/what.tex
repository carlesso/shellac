\begin{frame}

\frametitle{I protagonisti}

\begin{block}{Lo Shellac}
\`E interessante conoscere alcune caratteristiche di questi oggetti:
\begin{itemize}
\item[*]diametro del disco = 25 [cm.]
\item[*]massima escursione possibile del solco = $\sim$ 0.15 [mm.]
\item[*]banda utile = [30Hz. $\div$ 16000Hz.]
\end{itemize}
Il segnale è inciso lateralemente $\Rightarrow$ Uno scanner è 
sufficente per ricavare le forme d'onda
\end{block}

\begin{block}{Il Grammofono}
La riproduzione dei dischi \emph{shellac} avviene attraverso il \textbf{Grammofono}
che inizialmente prevedeva trasduzione da solco
a suono, meramente meccanica.

La diffusione dei grammofoni è andata scemando con l'avvento di nuovi supporti
e dei relativi strumenti di lettura e ciò li ha resi oggetti \textbf{costosi}
e \textbf{difficili da reperire}.
\end{block}

\end{frame}


